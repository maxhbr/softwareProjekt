\section{Faktorisierung von Polynomen über endlichen Körpern}
\texttt{GalFld/Core/Factorization.hs}\\
\texttt{GalFld/Algorithmen/Berlekamp.hs}\\
\texttt{GalFld/Algorithmen/Rabin.hs}


\paragraph{Faktorisierungsalgorithmen} Grundsätzlich sind alle Algorithmen, die
in \autoref{chap:algs} beschrieben werden und der konkreten Faktorisierung von
Polynomen dienen, als ħPolynom a -> [(Int, Polynom a)]ħ beschrieben. Das
bedeutet, dass eine \emph{Faktorisierung} stets als Liste von Tupeln zu
verstehen ist, wobei der erste Eintrag die Multiplizität des zweiten Eintrags,
dem konkreten Faktor, angibt.


\subsection{Triviale Faktoren} Kann man aus einem Polynom $f(X)$ 
den trivialen Faktor $X$ ausklammern, so leistet dies die Funktion
ħobviousFactorħ.

\haskellinput{Core/Factorization}{obviousFactor}

Um Polynome, die keinen trivialen Faktor besitzen trotzdem als Faktorisierung
zurückgegeben zu können, haben wir ħtoFactħ implementiert.

\haskellinput{Core/Factorization}{toFact}


\subsection{Funktionen rund um Faktorisierungen}

\paragraph{Anwenden von Faktorisierungen} 
Es ist klar, dass man verschiedene Algorithmen kombinieren will, die teilweise
Faktorisierungen herstellen (vgl. z.B. quadratfreie Faktorisierung mit
anschließendem Berlekamp). Dazu braucht man Funktionen, die einen
Faktorisierungsalgorithmus (also eine Funktion 
ħPolynom a -> [(Int,Polynom a)]ħ) auf eine bereits vorhandene 
Faktorisierung anwenden und anschließend das
Ergebnis zusammenfassen. Hierfür gibt es den nachstehenden Wrapper ħappFactħ.

\haskellinput{Core/Factorization}{appFact}

ħpotFactħ fasst dabei die entstehenden Mehrfachpotenzen der Faktoren zusammen.

\haskellinput{Core/Factorization}{potFact}

Es gilt zu bemerken, dass ħappFactħ parallelisiert ausgeführt wird, sofern die
Multicore-Unterstützung aktiviert wurde.

Man will jedoch als Anwender nicht immer ħappFactħ auf einen Algorithmus
anwenden. Daher gibt es für jeden Faktorisierungsalgorithmus eine Funktion 
ħappħ$A$ wobei $A$ für den jeweiligen konkreten Algorithmus steht.


\haskellinput{Core/Factorization}{appObFact}
\haskellinput{Algorithmen/SFreeFactorization}{appSff}
\haskellinput{Algorithmen/Berlekamp}{appBerlekamp}

Wie bereits erwähnt wird auf die konkreten Algorithmen erst in
\autoref{chap:algs} eingegangen.

\paragraph{Zusammenfassen von Faktorisierungen} Es ist sicherlich leicht
vorstellbar, dass durch Anwendung von ħappħ$A$ verschiedene Tupel entstehen,
deren eigentlicher Faktor jedoch der gleiche ist. 
ħaggFactħ ermöglicht die Zusammenfassung dieser Tupel nach Anwendung der 
Faktorisierungalgorithmen.

\haskellinput{Core/Factorization}{aggFact}

