\section{Weiteres}
\subsection{Die Klasse \texttt{ShowTex}}
Im Modul \texttt{GalFld.Core.ShowTex} wird eine Klasse \texttt{ShowTex}
implementiert, welche es entsprechenden Datentypen mit dieser Klasse
ermöglicht, nach \LaTeX{} zu rendern.

Die einzige zur Klasse gehörende Funktion ist \texttt{showTex}, welche einen
String mit \LaTeX{}-Code zurückgibt.

Ferner wurden Funktionen implementiert, die 
\begin{itemize}
  \item einen Datentyp der Klasse \texttt{ShowTex} bzw.
  \item einen String mit \LaTeX{}-Code in ein \texttt{PNG}-Bild umwandeln,
  \item sowie eine Funktion, die das Programm \texttt{sxiv} nutzt, 
    um die erzeugten \texttt{PNG}-Bilder anzuzeigen.
\end{itemize}
Diese drei Funktionen sind nur unter Linux verfügbar.

\haskellinput{Core/ShowTex}{renderTex}
\haskellinput{Core/ShowTex}{renderRawTex}

\subsection{Serialisierung}
Alle hier genutzten Datentypen haben eine Instanz vom Typ \texttt{Binary},
welche einfache Serialisierung ermöglicht. Dazu müssen nur die zwei Funktionen
ħputħ und ħgetħ implementiert werden.

Als Beispiel dient folgende Funktion, der ein Dateiname als
ħStringħ und eine Liste von Polynomen über einer Erweiterung von
einer Erweiterung von ħPFħ übergeben wird.
\begin{hcode}
import qualified Data.Binary as B

saveToFile :: String -> [Polynom (FFElem (FFElem PF))] -> IO ()
saveToFile fileName polys = writeFile fileName (B.encode (polys))
\end{hcode}

Die erzeugte Binärdatei lässt sich ebenso einfach wieder auslesen, 
indem man die
Daten in eine Variable ħrawħ lädt und diese mittels des Befehls
ħdecodeħ decodiert. Dabei muss der Typ der einzulesenden Rohdaten spezifiziert
werden.
\begin{hcode}
readFromFile :: String -> IO [Polynom (FFElem (FFElem PF))]
readFromFile fileName = do
  raw <- readFile fileName
  let ls = B.decode raw :: [Polynom (FFElem (FFElem PF))]
  return ls
\end{hcode}

\subsection{Spezielle Polynome und zahlentheoretische Funktionen}

\texttt{GalFld/More/SpecialPolys.hs}
\texttt{GalFld/More/NumberTheory.hs}

Vorgreifend auf \autoref{chap:bsp-prim-norm} wird hier der Inhalt von 
\texttt{SpecialPolys.hs} beschrieben. In dieser Datei wurden zwei spezielle
Familien von Polynomen über endlichen Körpern implementiert: die
Kreisteilungspolynome und die Pi-Polynome.

\subsubsection{Die Kreisteilungspolynome}

Die Kreisteilungspolynome lassen sich auf verschiedene Arten definieren. Hier
zitieren wir \cite[Abschnitt 4]{hach2013eki}.

\begin{definition}[Kreisteilungspolynom]
  Sei $\F_q$ ein endlicher Körper und $n\in \N$. Dann heißt für 
  $d\mid q^n-1$
  \[ \Phi_d(X) := \prod_{u \in C_d} (X-u)\]
  \emph{$d$-tes Kreisteilungspolynom}, wobei
  \[ C_d := \{ v\in \F_{q^n}^\ast:\ \ord(v) = d\}\]
  die Menge der $d$-ten primitiven Einheitswurzeln in $\F_q$ ist.%
  \footnote{Es sei vorausgesetzt, dass der Leser/indem Leser 
  die nicht explizit definierten Termini bekannt sind.}
\end{definition}

\begin{definition}[Möbius-Funktion]
  Seien $n \in \N^\ast$ und $n = \prod_{j=1}^l p_j^{a_j}$ seine
  Primfaktorzerlegung, so heißt
  \[ \mu(n) := \begin{cases} 1, &n=1\\ 0,& \exists j: a_j \geq 2\\
    (-1)^l, & a_j=1 \ \forall j \end{cases}\]
  \emph{Möbius-Funktion von $n$}.
\end{definition}

\begin{prop}
  Sei $d\in \N^\ast$. Dann gilt: 
  \[ \Phi_d(X) = \prod_{n\mid d} (X^n-1)^{\mu(\frac{d}{n})} \,.\]
\end{prop}
\begin{proof}
  \autocite[Abschnitt 4]{hach2013eki}.
\end{proof}

\paragraph{Implementierung}

Damit ist klar, wie man die Kreisteilungspolynome effizient implementiert.

\haskellinput{More/NumberTheory}{primFactors}
\haskellinput{More/NumberTheory}{isPrime}
\haskellinput{More/NumberTheory}{divisors}

\haskellinput{More/NumberTheory}{moebFkt}

\haskellinput{More/SpecialPolys}{cyclotomicPoly}

\subsubsection{Die Pi-Polynome}

\citeauthor{hach92} zeigt in \autocite{hach92}, wie man
\emph{alle} primitiven und normalen Elemente einer Körpererweiterung $\F_{q^n}$
über $\F_q$ als Nullstellen eines Polynoms finden kann.

\begin{bemerkung}
  \label{bem:primnorm}
  Wie man sich leicht überlegt, sind alle primitiven Elemente eines Körpers
  $\F_q$, also $u \in \F_q^\ast$ mit $\ord u = q-1$, gerade Nullstellen des
  $(q-1)$-ten Kreisteilungspolynoms. Auf ganz analoge Weise kann man die normalen
  Elemente, also diejenigen $u\in \F_{q^n}$, deren additive Ordnung 
  $\Ord_q(u)$ gleich $X^n-1$ ist, als Nullstellen der Pi-Polynome schreiben.
\end{bemerkung}

\begin{definition}[$q$-Polynom, {\autocite[Definition 1.2]{hach92}}]
  Sei $f(X) = \sum_{i=0}^n f_i X^i \in \F_q[X]$, so heißt 
  \[ F(X) := \sum_{i=0}^n f_i X^{q^i}\]
  das \emph{zu $f$ assoziierte $q$-Polynom}.
\end{definition}

\begin{definition}[Pi-Polynom, {\autocite[Definition 3.4]{hach92}}]
  Sei $f \in \F_q[X]$ ein monischer Teiler von $X^n-1$. Notiere
  \[ A_f := \{ u \in \F_{q^n}:\ \Ord(u) = f\}\,.\]
  Dann heißt 
  \[ P_f(X) := \prod_{v\in A_f} (X-v)\]
  das \emph{Pi-Polynom zu $f$ über $\F_q$}.
\end{definition}

\begin{definition}
  Für $f,g\in \F_q[X]$ definiere
  \[ (f\odot g)(X) := f(G(X))\,.\]
\end{definition}

Eine rekursiver Algorithmus zur Berechnung der Pi-Polynome ist dann nach
\autocite[Abschnitt 4]{hach92} gegeben durch:
\begin{pseudocode}{Berechnung Pi-Polynom}{alg:pipoly}
Input: $f(X) \in \F_q[X]$.
Output: $P_f(X) \in \F_q[X]$.
Algorithmus PIPOLY($f$):
  1. Berechne die vollständige Faktorisierung von $f$:
     $f(X) = \prod_{i=1}^k f_i^{\nu_i}$.
  2. Setze $P_{f_1} := F_1(X) X^{-1}$.
  3. Berechne $P_{f_1\ldots f_k}$ rekursiv durch
     $P_{f_1\ldots f_i} := (P_{f_1\ldots f_{i-1}} \odot f_i) P_{f_1\ldots f_{i-1}}^{-1}$
  4. Setze $P_f := P_{f_1\ldots f_k} \odot ( \prod_{i=1}^k f_i^{\nu_i-1})$
\end{pseudocode}

\paragraph{Implementierung} Man kann offenbar \autoref{alg:pipoly} direkt
in Haskell übertragen.

\haskellinput{More/SpecialPolys}{piPoly}

