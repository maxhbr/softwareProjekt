\chapter{Berechnete Werte} \label{Anh:Werte}
\section{Nicht-relativer Fall}
Die folgenden Tabellen sind nach Charakteristik der Körper sortiert. Diese
Charakteristik wird mit $p$ bezeichnet. In jeder Tabelle gibt es zu
verschiedenen Körpererweiterungen, welche mit $n$ bezeichnet werden, je eine
Zeile.
In jeder Zeile stehen nacheinander die Anzahl der primitiven Elemente, die
Anzahl der normalen Elemente und die Anzahl derer, die beide Eigenschaften
besitzen.

Im Kapitel \ref{Anh:quelle} ist der dazu verwendete Quellcode gelistet.

\foreach \x in {2,3,5,7,11,13,17,19,23,29,31,37,41,43,47,53,61,67,71,73,79,83,89,97,101} {
  \DTLloadrawdb{tab\x}{CSVs/CalcPrimNubers_p=\x.csv}
  \begin{table}[!htbp]
    \caption{Werte für $p=\x$}
    \DTLdisplaydb{tab\x}
  \end{table}
}

\section{Relativer Fall}
Hier wird der relative Fall, also eine Erweiterung von einer Erweiterung eines
Primkörpers betrachtet. Hier steht $m$ für den Grad der ersten Erweiterung.
\foreach \x in {2,3,4} {
  \DTLloadrawdb{tab2m\x}{CSVs/CalcPrimNubersRel_p=2_m=\x.csv}
  \begin{table}[!htbp]
    \caption{Werte für $p=2$ und $m=\x$}
    \DTLdisplaydb{tab2m\x}
  \end{table}
}

\pagebreak
\section{Der Quellcode} \label{Anh:quelle}
\subsection{Nicht-relativer Fall}
Hier das Bash-Script, das den Haskell Code generiert, der zur Berechnung der
Werte genutzt wurde. Das Script compiliert und startet die erzeugten Programme
danach parallel.
\lstinputlisting[language={},numbers=none,frame=L
                ,basicstyle=\footnotesize\ttfamily
                ,keywordstyle=\bfseries\color{black}
                ,commentstyle=\itshape\color{black}
                ,identifierstyle=\color{black}
                ,stringstyle=\color{black}
                ]{../calculation/cPN.sh}
\subsection{Relativer Fall}
\lstinputlisting[language={},numbers=none,frame=L
                ,basicstyle=\footnotesize\ttfamily
                ,keywordstyle=\bfseries\color{black}
                ,commentstyle=\itshape\color{black}
                ,identifierstyle=\color{black}
                ,stringstyle=\color{black}
                ]{../calculation/cPN.sh}
