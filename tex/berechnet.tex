\chapter{Berechnete Werte} \label{Anh:Werte}
Die folgenden Tabellen sind nach Charakteristik der Körper sortiert. Diese
Charakteristik wird mit $p$ bezeichnet. In jeder Tabelle gibt es zu
verschiedenen Körpererweiterungen, welche mit $n$ bezeichnet werden, je eine
Zeile.
In jeder Zeile stehen nacheinander die Anzahl der primitiven Elemente, die
Anzahl der normalen Elemente und die Anzahl derer, die beide Eigenschaften
besitzen.

Im nächstem Kapitel \ref{Anh:quelle} ist der dazu verwendete Quellcode
gelistet.

\foreach \x in {2,3,5,7,11,13,17,19,23,29,31,37,41,43,47,53,61,67,71,73,79,83,89,97,101} {
  \DTLloadrawdb{tab\x}{CSVs/CalcPrimNubers_p=\x.csv}
  \begin{table}[!htbp]
    \caption{Werte für $p=\x$}
    \DTLdisplaydb{tab\x}
  \end{table}
}

\section{Der Quellcode} \label{Anh:quelle}
Hier das Bash-Script, das den Haskell Code generiert, der zur Berechnung der
Werte genutzt wurde. Das Script compiliert und startet die erzeugten Programme
danach parallel.
\lstinputlisting[language={}]{../examples/cPN.sh}
