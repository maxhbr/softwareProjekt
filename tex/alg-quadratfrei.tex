Über endlichen Körpern existieren Über endlichen Körpern existieren 
verschiedene Ansätze, um ein Polynom zu faktorisieren. Diese sollen nun im
Folgenden erläutert werden.

\section{Quadratfreie Faktorisierung}
Wir beginnen mit der Beschreibung eines Algorithmus zur quadratfreien 
Faktorisierung. Dazu sei im Folgenden $k$ ein beliebiger Körper. 
Als Referenz dieses Abschnitts sei 
\autocite[Section 9]{cohan:algebra} und \autocite[Section
8.3]{geddes:algorithms} genannt.


\begin{definition}[quadratfrei, quadratfreier Teil]
  Seien $f(X) \in k[X]$ und seine vollständige Faktorisierung durch
  \[ f(X) = \prod_{i=1}^d f_i(X)^{\nu_i}\]
  gegeben. Der \emph{quadratfreie Teil von $f(X)$} ist
  \[ \nu(f(X)) = \prod_{i=1}^d f_i(X) \,.\]
  Ferner heißt $f(X)$ \emph{quadratfrei}, falls
  \[\nu(f(X)) = f(X)\,.\]
\end{definition}

\begin{definition}[quadratfreie Faktorisierung]
  Sei $f(X) \in k[X]$. Dann heißt
  \[ f(X) = c \prod_{i=1}^m r_i(X)^i\]
  \emph{quadratfreie Faktorisierung von $f(X)$}, falls
  für alle $i=1,\ldots,m$ gilt, dass $r_i(X)$ monisch und quadratfrei ist und
  für alle $i,j=1,\ldots,m$, $i\neq j$, stets $r_i(X)$ und $r_j(X)$ paarweise
  teilerfremd sind.
\end{definition}

Bekanntlich ist für jedes nichttriviale Polynom $f(X)$ über einem Körper der
Charakteristik Null $\ggT(f(X),f'(X)) \neq 0$, wobei $f'(X)$ die formale
Ableitung von $f$ meint. Damit kann man folgern, dass $\ggT(f(X),f'(X)) = 1$
genau dann, wenn $f(X)$ quadratfrei ist. (vgl. \autocites[Theorem
9.4]{cohan:algebra}[Theorem 9.5]{cohan:algebra}) Über endlichen Körpern geht
dies nicht so einfach, wie folgendes Beispiel zeigt:

\begin{beispiel}
  Sei $f(X) = X^3+1 \in \F_3[X]$. Dann ist 
  \[ f'(X) = 3 X^2 = 0 \,. \]
  Dennoch besitzt $f(X)$ eine quadratfreie Faktorisierung, da
  \[ f(X) = (x+1)^3\,.\]
\end{beispiel}


\subsection{Algorithmus zur quadratfreien Faktorisierung über endlichen
Körpern}
Einen Algorithmus zur quadratfreien Faktorisierung über Körpern der
Charakteristik $0$ findet man beispielsweise in \autocite[Figure 9.1]{cohen:algebra}
oder \autocite[Algorithm 8.1]{geddes:algorithms}.

Für den passenden Algorithmus über endlichen Körpern halten wir uns an
\autocite[Section 8.3]{geddes:algorithms}. Dazu starten wir mit der
wesentlichen Aussage, die gerade in dem Fall, dass die Ableitung eines Polynoms
$0$ ist, die entscheidende Information liefert. Doch dies gilt nicht für
beliebige Körper!

\begin{definition}[perfekter Körper]
  Ein Körper $\F$ heißt \emph{perfekt}, falls $\charak \F = p$ für eine
  Primzahl $p$ und der Frobenius $\sigma: \F \to \F,\ x \mapsto x^p$ ein
  Automorphismus ist.
\end{definition}

\begin{prop}\label{prop:sfree1}
  Seien $\F$ ein perfekter Körper und $f(X) \in \F[X]$. 
  Ist $f'(X) = 0$, so existiert ein $b(X) \in \F[X]$ mit
  \[ f(X) = (b(X))^p\,.\]
\end{prop}

\begin{proof}
  Sei $f$ gegeben als $f(X) = a_nX^n + \ldots + a_0$, so gilt offensichtlich
  durch Betrachtung der Definition und Regeln der formalen Ableitung, dass
  jede auftauchende Potenz von $X$ ein Vielfaches von $p$ sein muss. Also ist
  \[ f(X) = a_{pk} X^{pk} + \ldots + a_p X^p + b_0\,.\]
  Definiere nun 
  \[ b(X) = b_k X^k + \ldots + b_1 X + b_0 \quad\text{mit}\quad
    b_i = a_{pi}^{\frac 1 p}\ i=0,\ldots,k\,.\]
  Da wir wissen, dass der Frobenius $\F \to \F, x \mapsto x^p$ ein
  Automorphismus auf $\F$ ist, ist $(.)^{\frac 1 p}$ ein wohldefinierter
  Ausdruck und es gilt 
  \[ f(X) = b(X)^p\,.\]
\end{proof}

\begin{beispiel}
  Sei $F = \F_p(z)$ für $z\notin \F_p$. Dann gilt für $f(X) = X^p-z \in F[X]$,
  dass $f'(X) = 0$, aber $f(X)$ ist über $F$ irreduzibel. Offenbar besitzt $z$
  kein Urbild unter dem Frobenius; mithin ist $F$ nicht perfekt.
\end{beispiel}


Damit können wir nun einen Algorithmus zur quadratfreien Faktorisierung über
endlichen Körpern formulieren.


\begin{pseudocode}{Quadratfreie Faktorisierung über endlichen Körpern}%
 {alg:sfree}
Input:  $f(X) \in \F_q[X]$ monisch, $q=p^n$ eine Primzahlpotenz.
Output: $f(X) = r(X) = \prod_{i=1}^m r_i(X)^i$ quadratfreie Faktorisierung
Algorithmus SFF($f(X)$):
$i:=1$, $r(X) := 1$, $b(X) := f'(X)$.
if $b(X) \neq 0$ then		(1)
  $c(X) := \ggT(f(X),b(X))$
  $w(X) := f(X) / c(X)$
  while $w(X) \neq 1$ do		(1.1)
    $y(X) := \ggT(w(X),c(X))$, $z(X) := w(X) / y(X)$
    $r(X) := r(X)\cdot z(X)^i$, $i := i+1$
    $w(X) := y(X)$, $c(X) := c(X) / y(X)$
  endwhile
  if $c(X) \neq 1$ then		(1.2)
    $c(X) := c(X)^{\frac 1 p}$
    $r(X) := r(X) \cdot ($SFF($c(X)$)$)^p$
  endif
else		(2)
  $f(X) := f(X)^{\frac 1 p}$
  $r(X) := ($SFF($f(X)$)$)^p$
endif
\end{pseudocode}


\begin{thm}
  \autoref{alg:sfree} berechnet die quadratfreie Faktorisierung für Polynome
  über endlichen Körpern (sogar über perfekten Körpern).
\end{thm}
\begin{proof}
  Sei $f(X)$ gegeben durch seine vollständige Faktorisierung in irreduzible
  Faktoren 
  \[ f(X) = \prod_{i=1}^d f_i(X)^{\nu_i}\]
  \begin{description}
    \item[Schritt 2.] Beginnen wir mit dem kürzeren Fall.
      Ist $b(X) = f'(X) = 0$, so existiert nach \thref{prop:sfree1} eine
      $p$-te Wurzel des Polynoms. Auf diese lässt sich dann der Algorithmus
      rekursiv anwenden.
    \item[Schritt 1.] Kommen wir nun zu dem Fall, wo auch wirklich etwas zu tun
      ist. Sei zunächst eine quadratfreie Faktorisierung von $f(X)$ gegeben,
      d.h.
      \[ f(X) = \prod_{i=1}^m a_i(X)^i \]
      mit $a_i$ quadratfrei und paarweise teilerfremd.
      Nun ist
      \[ b(X) = f'(X) = \sum_{i=1}^m a_1(X) \cdot \ldots\cdot a_{i-1}(X)^{i-1}
        \cdot i a_i(X)^{i-1} a_i'(X)\cdot a_{i+1}(X)^{i+1}\cdot\ldots\cdot
        a_m(X)^m\]
      und wir können folgern, 
      \[ c(X) = \ggT(f(X),f'(X)) = \prod_{i\in A} a_i(X)^{i-1}\,,\]
      wobei $A = \{i=1,\ldots,m:\ p\nmid i\}$. Es ist klar, dass diejenigen
      $a_i$, deren Exponent $i$ ein Vielfaches der Charakteristik ist,
      nicht mehr im $\ggT$ auftauchen. Damit haben wir
      \[ w(X) = \frac{f(X)}{c(X)} = \prod_{i\in A} a_i(X) \]
      ein Produkt der quadratfreien Faktoren in $A$ mit jeweils einfacher
      Vielfachheit. Dieses können wir nun nutzen, um diese quadratfreien Faktoren
      zu isolieren: Für $A = \{i_1,\ldots,i_k\}$ in aufsteigend sortierter
      Reihenfolge haben wir
      \begin{align*}
        y(X) &= \ggT(w(X),c(X)) = \prod_{j=i_1}^{i_k} a_j(X)\,,\\
        z(X) &= \frac{w(X)}{y(X)} = a_{i_1}(X)\,.
      \end{align*}
      Nun ist klar, dass man die weiteren Faktoren deren Exponenten in $A$
      liegen durch iterative Anwendung dieser Idee erhält, wie man im 
      Algorithmus erkennen kann. Letztlich bleibt nur die Frage, wie man an die
      Faktoren kommt, deren Exponenten Vielfache der Charakteristik sind.
      Dies ist aber auch offensichtlich, betrachtet man erneut
      \thref{prop:sfree1} und die Umsetzung in Schritt 1.2. 
  \end{description}
\end{proof}
