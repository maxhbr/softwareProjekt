Über endlichen Körpern existieren Über endlichen Körpern existieren 
verschiedene Ansätze, um ein Polynom zu faktorisieren. Diese sollen nun im
Folgenden erläutert werden.

\section{Quadratfreie Faktorisierung}
Wir beginnen mit der Beschreibung eines Algorithmus zur quadratfreien 
Faktorisierung. Dazu sei im Folgenden $k$ ein beliebiger Körper. 
Als Referenz dieses Abschnitts sei 
\autocite[Section 9]{cohan:algebra} und \autocite[Section
8.3]{geddes:algorithms} genannt.


\begin{definition}[quadratfrei]
  Sei $f(X) \in k[X]$. Dann heißt $f(X)$ \emph{quadratfrei},
  falls für seine vollständige Faktorisierung
  \[ f(X) = \prod_{i=1}^d f_i(X)^{\nu_i}\,,\]
  $f_i(X)$ irreduzibel und paarweise verschieden, gilt:
  \[ \nu_i = 1 \ \forall i=1,\ldots,d\,.\]
\end{definition}

\begin{definition}[quadratfreie Faktorisierung]
  Sei $f(X) \in k[X]$. Dann heißt
  \[ f(X) = c \prod_{i=1}^m r_i(X)^i\]
  \emph{quadratfreie Faktorisierung von $f(X)$}, falls
  für alle $i=1,\ldots,m$ gilt, dass $r_i(X)$ monisch und quadratfrei ist und
  für alle $i,j=1,\ldots,m$, $i\neq j$, stets $r_i(X)$ und $r_j(X)$ paarweise
  teilerfremd sind.
\end{definition}

Bekanntlich ist für jedes nichttriviale Polynom $f(X)$ über einem Körper der
Charakteristik Null $\ggT(f(X),f'(X)) \neq 0$, wobei $f'(X)$ die formale
Ableitung von $f$ meint. Damit kann man folgern, dass $\ggT(f(X),f'(X)) = 1$
genau dann, wenn $f(X)$ quadratfrei ist. (vgl. \autocites[Theorem
9.4]{cohan:algebra}[Theorem 9.5]{cohan:algebra}) Über endlichen Körpern geht
dies nicht so einfach, wie folgendes Beispiel zeigt:

\begin{beispiel}
  Sei $f(X) = X^3+1 \in \F_3[X]$. Dann ist 
  \[ f'(X) = 3 X^2 = 0 \,. \]
  Dennoch besitzt $f(X)$ eine quadratfreie Faktorisierung, da
  \[ f(X) = (x+1)^3\,.\]
\end{beispiel}


\subsection{Algorithmus zur quadratfreien Faktorisierung über endlichen
Körpern}
Einen Algorithmus zur quadratfreien Faktorisierung über Körpern der
Charakteristik $0$ findet man beispielsweise in \autocite[Figure 9.1]{cohen:algebra}
oder \autocite[Algorithm 8.2]{geddes:algorithms}.

Für den passenden Algorithmus über endlichen Körpern halten wir uns an
\autocite[Section 8.3]{geddes:algorithms}. Dazu starten wir mit der
wesentlichen Aussage, die gerade in dem Fall, dass die Ableitung eines Polynoms
$0$ ist, die entscheidende Information liefert.

Im Folgenden sei $\F$ ein Körper der Charakteristik $p$ für $p$ eine Primzahl.

\begin{prop}
  Sei $f(X) \in \F[X]$. Ist $f'(X) = 0$, so existiert ein $b(X) \in \F[X]$ mit
  \[ f(X) = (b(X))^p\,.\]
\end{prop}

\begin{proof}
  Sei $f$ gegeben als $f(X) = a_nX^n + \ldots + a_0$, so gilt offensichtlich
  durch Betrachtung der Definition und Regeln der formalen Ableitung, dass
  jede auftauchende Potenz von $X$ ein Vielfaches von $p$ sein muss. Also ist
  \[ f(X) = a_{pk} X^{pk} + \ldots + a_p X^p + b_0\,.\]
  Definiere nun 
  \[ b(X) = b_k X^k + \ldots + b_1 X + b_0 \quad\text{mit}\quad
    b_i = a_{pi}^{\tfrac 1 p}\ i=0,\ldots,k\,.\]
  Da wir wissen, dass der Frobenius $\F \to \F, x \mapsto x^p$ ein
  Automorphismus auf $\F$ ist, ist $(.)^{\tfrac 1 p}$ ein wohldefinierter
  Ausdruck und es gilt 
  \[ f(X) = b(X)^p\,.\]
\end{proof}


Damit können wir nun einen Algorithmus zur quadratfreien Faktorisierung über
endlichen Körpern formulieren.


\begin{pseudocode}{Quadratfreie Faktorisierung über endlichen Körpern}%
 {alg:sfree}
Input:  $f(X) \in \F_q[X]$ monisch, $q=p^m$ eine Primzahlpotenz.
Output: $f(X) = \prod_{i=1}^d r_i(X)^d$
Algorithmus SFF($f(X)$):
 1. Setze $i:=1$, $r(X) := 1$, $b(X) := f'(X)$.
 2. if $b(X) \neq 0$ then
    $c(X) := \ggT(f(X),b(X))$.
    $w(X) := f(X) / 
\end{pseudocode}



