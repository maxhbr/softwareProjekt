\section{Der Algorithmus von Berlekamp}
\label{sec:berlekamp}
Sei im Folgenden $\F_q$ ein endlicher Körper von Charakteristik $p$ und
$f(X) \in \F_q[X]$ monisch. Ziel dieses Abschnittes ist ein Algorithmus, der eine
vollständige Faktorisierung von $f(X)$ über $\F_q$ angibt.

\subsection{Idee}
Die grundlegende Idee des Berlekamp-Algorithmus besteht in folgendem Lemma.

\begin{lemma}\label{lemma:berlekamp1}
  Es gilt
  \[ X^q - X = \prod_{a\in \F_q} (X- a) \in \F_q[X] \,.\]
\end{lemma}
\begin{proof}
 \autocite[Theorem 6.1 mit Corollary 4.5]{wan2003lectures}.
\end{proof}

Ersetzen wir $X$ durch ein Polynom $g(X) \in \F_q[X]$, so erhalten wir
\[ g(X)^q - g(X) = \prod_{a\in \F_q} (g(X) - a)\]
und können uns nun überlegen, falls wir ein Polynom $g(X)$ mit
$g(X)^q - g(X) \bmod f(X)$ haben, dann
\[ f(X) \mid \prod_{a\in \F_q} (g(X)-a) \,,\]
was zumindest eine teilweise Faktorisierung von $f(X)$ liefern könnte.

Dass dies in der Tat funktioniert, zeigt nachstehender Satz, der obige
Idee zusammenfasst und konkretisiert.

%\begin{lemma}
  %Sei $g(X) \in \F_q[X]$. Dann ist äquivalent:
  %\begin{enumerate}
    %\item $g(X)^q \equiv g(X) \bmod f(X)$
    %\item $f(X) \mid \prod_{a \in \F_q} (g(X) -a)$
    %\item $f(X) \mid \prod_{a\in \F_q} \ggT(f(X), g(X)-a)$
  %\end{enumerate}
%\end{lemma}
%\begin{proof}
  %\autocite[Lemma 9.3]{wan2003lectures}.
%\end{proof}

\begin{thm}
  \label{satz:berlekamp1}
  Sei $g(X) \in \F_q[X]$ mit
  \[ g(X)^q \equiv g(X)  \bmod f(X)\,,\]
  so gilt
  \[ f(X) = \prod_{a\in \F_q} \ggT(f(X),g(X)-a)\]
  und die $\ggT$s sind paarweise teilerfremd.
\end{thm}
\begin{proof}
  \autocite[Theorem 9.1]{wan2003lectures}.
\end{proof}


\subsection{Die Berlekamp-Algebra}

Damit haben wir nun eine Motivation für folgende Definition.

\begin{definition}
  Der \emph{Berlekamp-Raum zu $f(X)$} bzw. die \emph{Berlekamp-Algebra zu
  $f(X)$} ist
  \[ \cB_f := \{ h(X) \in \F_q[X]:\ \deg h < \deg f, h(X)^q \equiv h(X) \bmod
    f(X)\}\,.\]
\end{definition}

\begin{bemerkung}
  In der Tat wird $\cB_f$ zu einer $\F_q[X]\big/(f(X))$-Algebra.
\end{bemerkung}

Nun können wir ein wesentliches Resultat zitieren:

\begin{thm}
  \label{satz:berlekamp2}
  Es gilt:
  \[ \dim_F(\cB_f) = s\,,\]
  wobei $s$ die Anzahl irreduzibler paarweise verschiedener Faktoren von $f(X)$
  ist.
\end{thm}
\begin{proof}
  \autocite[Satz 6.2]{hach2013ek}.
\end{proof}

Nun stellt sich natürlich die Frage, wie man konkret Elemente aus der
Berlekamp-Algebra zu $f(X)$ findet. Blickt man noch einmal auf die Definition
von $\cB_f$, so erkennt man, dass $\cB_f$ gerade der Kern folgender linearen
Abbildung ist:
\[ \Gamma_f:\ \funcdef{
  \F_q[X]_{<\deg f} &\to& \F_q[X]_{<\deg f},\\
  g(X) &\mapsto& g(X)^q - g(X) \bmod f(X)\,.}\]
Nun können wir aber eine Darstellungsmatrix von $\Gamma_f$ angeben, da wir
die kanonische Basis von $\F_q[X]_{<\deg f}$ angeben können durch
\[ \{ 1, X, X^2, \ldots, X^{\deg f -1}\}\,. \]

\subsection{Der Berlekamp-Algorithmus}
Eine Kleinigkeit fehlt obigem Vorgehen noch, um daraus sicher eine teilweise
Faktorisierung von $f(X)$ gewinnen zu können. Es ist a priori nicht klar, dass
in \thref{satz:berlekamp1} die auftretenden $\ggT$s eine echte, also nicht
degenerierte, Faktorisierung von $f(X)$ liefern. Doch dies ist offensichtlich,
falls $g(X) \in \cB_f \setminus \F_q$ gewählt wird. Dann ist $\deg g < \deg f$
und daher $\ggT(f(X),g(X)-a) \neq f(X)$ für alle $a \in \F_q$.

Letztlich liefert noch nachstehendes Lemma die Grundlage für eine rekursive
Anwendung des Algorithmus:

\begin{lemma}
  \label{lemma:berlekamp3}
  Ist $a(X)\in \F_q[X]$ monisch mit $a(X) \mid f(X)$, so ist
  \[ \pi: \funcdef{ \cB_f &\to& \cB_a \\ g(X) &\mapsto& g(X) \bmod a(X)} \]
  eine surjektive lineare Abbildung.
\end{lemma}
\begin{proof}
  klar.
\end{proof}

\begin{comment}
  Ist $\pi$ nicht sogar ein $\F_q[X]/(a(X))$-Algebrenhomomorphismus??
\end{comment}


\subsubsection{Implementierung}
Um tatsächlich eine \emph{vollständige} Faktorisierung zu erhalten, muss man
sich noch überlegen, dass dies mit Hilfe des Berlekamp-Algorithmus nur möglich
ist, falls $f(X)$ quadratfrei ist (vgl. \thref{satz:berlekamp2}!). Daher sei
im Folgenden $f(X)$ stets ein quadratfreies, monisches Polynom über $\F_q[X]$.

\paragraph{Berechnung einer Basis von $\cB_f$} Die Basis des Berlekampraumes
berechnen wir mit den in \autoref{sec:linalg} vorgestellten Methoden der
linearen Algebra.

\haskellinput{Algorithmen/Berlekamp}{berlekampBasis}

Die Funktion ħred iħ liefert dabei gerade das ħiħ-te Basiselement der
kanonischen Basis von $\F_q[X]\big/(f(X))$.

\paragraph{Der Berlekamp-Algorithmus} Bei einer konkreten Umsetzung des
Berlekamp-Algorithmus bleibt immer die Frage, wie ein Element $g(X)\in
\cB_f\setminus \F_q$ zu wählen ist. Wir haben uns entschieden, stets das zweite
Basiselement (das erste ist immer $1$) zu wählen. Sicherlich könnte man auch
ein zufälliges Element wählen; dies widerspricht aber des funktionalen
Prinzips, das Haskell zugrunde liegt.

\haskellinput{Algorithmen/Berlekamp}{berlekampFactor}

Die Berechnung der neuen Basis bei der rekursiven Anwendung ist aufgrund
\thref{lemma:berlekamp3} relativ einfach, da $\pi$ simplerweise auf die schon
vorhandene Berlekampbasis angewendet werden kann.

\begin{beispiel}
  Angenommen wir wollen das Polynom
  \[ X^5 + X^4 + 3 X^3 + 3 X^2 + 2 X + 2 \ \in \F_5[X]\]
  faktorisieren. Wir berechnen also
  \[ \begin{array}{r@{\ \equiv\ }l@{\ \bmod f(X)}}
    1^5 & 1\\
    X^5 & 4X^4 + 2X^3 + 2 X^2+ 3X + 3\\
    X^{10} & X^2\\
    X^{15} & 2X^4 + 2X^3 + X^2 + X + 4\\
    X^{20} & X^4
  \end{array} \]
  und erhalten damit eine Darstellungsmatrix von $\Gamma$ bezüglich der Basis
  $\{ 1, X, X^2, X^3, X^4\}$ von $\F_5[X]_{<5}$ und können diese in
  Zeilenstufenform bringen:
  \[ D_\Gamma = \begin{bmatrix}
      1& 0& 0& 0& 0\\
      3& 3& 2& 2& 4\\
      0& 0& 1& 0& 0\\
      4& 1& 1& 2& 2\\
      0& 0& 0& 0& 1\end{bmatrix} \rightsquigarrow
      \begin{bmatrix}
      1& 0& 0& 0& 0\\
      0& 1& 0& 3& 0\\
      0& 0& 1& 0& 0\\
      0& 0& 0& 0& 1\\
      0& 0& 0& 0& 0\end{bmatrix}\,. \]
  Also ist eine Basis von $\cB_f$ gegeben durch
  \[ B_f := \{ 1, X^3 +X, X^2, X^4\}\,.\]
  Wir wählen -- wie oben beschrieben -- das zweite Basiselement $h(X) = X^3 +X$
  aus und berechnen
  \[ \begin{array}{r|lllll}
      a\in \F_q & 0 & 1 & 2 & 3 & 4 \\\hline
      \ggT(f(X), h(X)-a) & X^2+2 & X^2+4X+3 & 1 & 1 & X+2\end{array}\,.\]
  Dies erlaubt nun iterative Anwendung des Berlekamp-Algorithmus, nämlich für
  $X^2+2$, $X^2+4X+3$ und für $X+2$. Letzteres ist natürlich offensichtlich
  irreduzibel.

  Für $f_1(X) := X^2+2$ haben wir
  \[ B_f \bmod f_1(X) = \{ 1, 0, 3, 4\}\]
  und für $f_2(X) := X^2+4X+3$
  \[ B_f \bmod f_2(X) = \{1,1,X+2,1\}\,.\]
  Für $f_1$ bricht der Berlekamp-Algorithmus sofort ab, da offenbar die
  Dimension des Berlekampraumes $1$ ist.
  Für $f_2$ wählen wir $h(X) = X+2$ und erhalten
  \[ \begin{array}{r|lllll}
      a\in \F_q & 0 & 1 & 2 & 3 & 4 \\\hline
      \ggT(f_2(X), h(X)-a) & 1 & 1 & X+3 & 1 & X+1\end{array}\,.\]
  Damit ist die vollständige Faktorisierung von $f(X)$ über $\F_5$ bekannt:
  \[ f(X) = (X+1)(X+2)(X+3)(X^2+2)\,.\]
\end{beispiel}
