\begin{titlepage}
\thispagestyle{empty}
\newcommand{\Rule}{\rule{\textwidth}{1mm}}
\begin{center}\sffamily\bfseries
\LARGE\textcolor{gray}{Softwareprojekt}
\par\vspace*{1cm}
\tikz[baseline]{ \node[anchor=base, minimum width=\textwidth,
  inner xsep=5pt,
  inner ysep=10pt,
  align=center,
  text width=0.8\textwidth,
  font=\Huge]
  (main title 1) {GalFld};
\node[anchor=base, minimum width=\textwidth,
  inner xsep=5pt,
  inner ysep=10pt,
  align=center,
  text width=\textwidth,
  font=\Large,
  below=10pt of main title 1.south]
  (main title) {%
    Eine Umsetzung endlicher Körper \\
    in der Programmiersprache Haskell\\
    mit besonderer Betrachtung der \\
    Faktorisierung von Polynomen über endlichen
    Körpern};
  \draw[overlay, line width=1mm, gray,
    line cap=round]
    (main title.south west)
    ++(0,-10pt) -- +(\textwidth,0)
    (main title 1.north west)
    ++(0,10pt) -- +(\textwidth,0);
  \draw[overlay, line width=1pt, gray,
    line cap=round]
    (main title 1.south west)
    ++(2cm,-5pt) -- +($(\textwidth,0)+(-4cm,0)$);
}
\vfill
\normalfont\sffamily\large vorgelegt von\par
\bfseries\LARGE Stefan Hackenberg und Maximilian Huber
\vfill
\normalfont\sffamily\large am\\
\bfseries\Large Institut für Mathematik\\
\normalfont\sffamily\large der\\
\bfseries\Large Universität Augsburg
\vfill
\normalfont\sffamily\large betreut durch \\
\bfseries\Large Prof. Dr. Dirk Hachenberger\par
\vfill
\normalfont\sffamily\large abgegeben am\\
\bfseries\Large 09.09.2014
\end{center}
\end{titlepage}

