\section{Endliche Körper}
\subsection{Primkörper}
Die Primkörper werden in dem Modul \texttt{GalFld.Core.PrimeFields}
spezifiziert. Diese werden implementiert als ħIntħ Werte, die durch den Wrapper
ħModħ noch zusätzlich die Information enthalten, in welchem 
Primkörper sich das Element befindet.

Da wir die Charakteristik zu einem solchem Körper auf Typebene speichern
wollen, führen wir zunächst eine neue Typklasse mit dem Namen
ħNumeralħ ein, welche als einzige Funktion 
ħnumValue :: a -> Intħ besitzen.
Diese Funktion soll konstant die Charakteristik wiedergeben.

Nun können wir durch
\haskellinput{Core/PrimeFields}{newtype Mod}
Primkörper definieren, wobei für den Parameter \texttt{n} ein
Datentyp von der Klasse \texttt{Numeral} eingesetzt werden soll.

Um zu einem Element im Primkörper einen Repräsentanten in $\Z$ zu erhalten, gibt
es die Funktion
\haskellinput{Core/PrimeFields}{unMod}
Einen Repräsentanten, der nichtnegativ, aber kleiner als die Charakteristik ist,
liefert
\haskellinput{Core/PrimeFields}{getRepr}

Die Instanzen von ħShowħ und ħShowTexħ ermöglichen es, Elemente von Primkörpern
als String oder als \LaTeX{}-Code darzustellen.
\haskellinput{Core/PrimeFields}{instance .*?=> Show}
\haskellinput{Core/PrimeFields}{instance .*?=> ShowTex}

Ferner implementieren wir Instanzen von ħEqħ, ħNumħ und ħFiniteFieldħ.
\haskellinput{Core/PrimeFields}{instance .*?=> Eq}
\haskellinput{Core/PrimeFields}{instance .*?=> Num}
\haskellinput{Core/PrimeFields}{instance .*?=> FiniteField}
\haskellinput{Core/PrimeFields}{modulus}

Zum bequemen Invertieren haben wir auch noch eine Instanz von ħFractionalħ
hinzugefügt.
\haskellinput{Core/PrimeFields}{instance .*?=> Fractional}

Weiterhin haben wir noch die folgenden Instanzen:
\haskellinput{Core/PrimeFields}{instance .*?=> Binary}
\haskellinput{Core/PrimeFields}{instance .*?=> NFData}

\subsubsection{Erzeugen von Primkörpern}
Möchte man nun einen Primkörper von beliebiger Charakteristik in einem 
Haskell- Programm benutzen,
bietet sich die ħTemplateHaskellħ Funktion ħgenPrimeFieldħ an. Diese
übernimmt das Erzeugen von diversen Instanzen, die nötig sind.
\haskellinput{Core/PrimeFields}{genPrimeField}

Da es sich hierbei um eine Funktion handelt, die per ħTemplateHaskellħ zur
Compilezeit ausgeführt wird, sind die beiden folgenden Pragmas nötig:
\begin{hcode}
{-# LANGUAGE QuasiQuotes #-}
{-# LANGUAGE TemplateHaskell #-}
\end{hcode}
Dann kann beispielsweise der Primkörper der Charakteristik $7$ namens
\texttt{PF} durch 
\begin{hcode}
$(genPrimeField 7 "PF")
\end{hcode}
%%%%%%%%%%%%%%%%%%%%%%%%%%%%%%%%%%%%%%%%%%%%%%%%%%%%%%%%%%%%%%%%%%%%%%%%%%%%%%%%%%%%%%%%%%%%%%%%%%%%%%%%%%%%%%%%%%%%%%%%%%%%%%%%%
erzeugt werden.

\subsection{Erweiterungskörper}
Um Erweiterungskörper darzustellen, verwenden wir Polynome, welche modulo einem
Minimalpolynom gelesen werden sollen; codiert in dem Datentyp
\texttt{FFElem}.
\haskellinput{Core/FiniteFields}{data FFElem}

Durch dieses Konzept kann man einfach in Erweiterungen von Erweiterungen
rechnen.
Startet man mit einem Primkörper, beispielsweise $\F_2$, so haben wir
darin das Element $1$:
\begin{hcode}
f2 = 1::F2
\end{hcode}
Durch das Minimalpolynom $X^2+X+1$ erzeugen wir uns eine Erweiterung vom 
Grad 2:
\begin{hcode}
e2f2Mipo = pList [1::F2,1,1] -- x^2+x+1
e2f2 = FFElem (pList [0,1::F2]) e2f2Mipo
\end{hcode}
Hier ist \texttt{e2f2} ein erzeugendes Element. Dies reicht, 
um alle Körperelemente generieren zu können.
Durch eine weitere Grad $2$ Erweiterung erhalten wir:
\begin{hcode}
e2e2f2Mipo = pList [e2f2,one,one] -- x^2+x+e2f2
e2e2f2 = FFElem (pList [0,e2f2]) e2e2f2Mipo
\end{hcode}
Alternativ kann man auch durch eine Grad $4$ Erweiterung über $\F_2$ den
gleichen Körper erhalten:
\begin{hcode}
e4f2Mipo = pList [1::F2,1::F2,0,0,1::F2] -- x^4+x^2+1
e4f2 = FFElem (pList [0,1::F2]) e4f2Mipo
\end{hcode}

Öffnet man \texttt{GalFld.Sandbox.FFSandbox} mit \texttt{GHCI} startet der
Interpreter und man befindet sich in einer Umgebung, in der die Körper bereits
erzeugt wurden. Nachdem wir also bereits das Element \texttt{e2e2f2} haben,
können wir uns dieses anzeigen lassen, indem wir einfach nur \texttt{e2e2f2} in
die Konsole eintippen und bestätigen. Damit erhalten wir
\begin{lstlisting}[language=bash
                  ,numbers=none
                  ,frame=L]
((1₂·X mod 1₂·X²+1₂·X+1₂)·X mod (1₂ mod ...)·X²+(1₂ mod ...)·X+(1₂·X mod 1₂·X²+1₂·X+1₂))
\end{lstlisting}
Ein Element in einer Körpererweiterung wird beispielsweise dargestellt als
\begin{itemize}
  \item \texttt{(1₂·X mod 1₂·X²+1₂·X+1₂)}, welches die Äquivalenzklasse von $x$
  in $\F_2[X]/(X^2+X+1)$ bezeichnet. Die \LaTeX{} Darstellung dazu ist 
  $\left(
      \underline{
        1_{2}
        \cdot{}X
      }_{
        mod~1_{2} \cdot{}X^{2}+1_{2}\cdot{}X+1_{2}
      }
    \right)$.
  \item \texttt{(1₂ mod ...)} bedeutet, dass noch nicht klar ist, modulo
  welchem Polynom dieses Element gelesen wird. Es ist also die
  $1\in\F_2[x]/(g(x))$
  wobei $g(x)$ erst während der Berechnung inferiert werden muss.
  Dies ist nötig, um die Inklusion des Grundkörpers zu realisieren.
\end{itemize}
Durch die ħShowTexħ-Instanz können wir ħe2e2f2ħ auch in \LaTeX{} darstellen:
\[
\left(
  \underline{
    \left(
      \underline{
        1_{2}
        \cdot{}X
      }_{
        mod~1_{2} \cdot{}X^{2}+1_{2}\cdot{}X+1_{2}
      }
    \right)\cdot{}X
  }_{
    mod~1_{2}\cdot{}X^{2}+1_{2}\cdot{}X+\left(
      \underline{
        1_{2}\cdot{}X
      }_{
        mod~1_{2}\cdot{}X^{2}+1_{2}\cdot{}X+1_{2}
      }
    \right)
  }
\right)
\]
Ersetzen wir \texttt{(1₂·X mod 1₂·X²+1₂·X+1₂)} mit \texttt{Y} dann kann
\texttt{e2e2f2} auch geschrieben werden als:
\begin{lstlisting}[language=bash
                  ,numbers=none
                  ,frame=L]
(Y·X mod (1₂ mod ...)·X²+(1₂ mod ...)·X+Y)
\end{lstlisting}
Dieses Element ist also die Äquivalenzklasse von $YX$ in
$\F_2[Y,X]/(Y^2+Y+1,X^2+X+Y)$.

Nun können wir z.B.
ħe2e2f2 + e2e2f2 * e2e2f2ħ berechnen und erhalten:
\begin{lstlisting}[language=bash
                  ,numbers=none
                  ,frame=L]
((1₂ mod 1₂·X²+1₂·X+1₂)·X+(1₂ mod 1₂·X²+1₂·X+1₂) mod (1₂ mod ...)·X²+
(1₂ mod ...) ·X+(1₂·X mod 1₂·X²+1₂·X+1₂))
\end{lstlisting}
In \LaTeX{}:
\[
\left(
  \underline{
    \left(
      \underline{
        1_{2}
      }_{
        mod~1_{2}\cdot{}X^{2}+1_{2}\cdot{}X+1_{2}
      }
    \right)
    \cdot{}X+\left(
      \underline{
        -1_{2}
      }_{
        mod~1_{2}\cdot{}X^{2}+1_{2}\cdot{}X+1_{2}
      }
    \right)
  }_{
    mod~1_{2}\cdot{}X^{2}+1_{2}\cdot{}X+\left(
      \underline{
        1_{2}\cdot{}X
      }_{
        mod~1_{2}\cdot{}X^{2}+1_{2}\cdot{}X+1_{2}
      }
    \right)
  }
  \right)
\]

\subsubsection{Funktionen auf Körpererweiterungen}
Wie wir gesehen haben, ist der zugrunde liegende Körper nicht bei jedem
Koeffizienten eines Polynoms klar. Daher liefert ħgetReprPħ für ein Polynom
einen passenden Repräsentanten und ħcharOfPħ als abkürzende Schreibweise dessen
Charakteristik.
\haskellinput{Core/FiniteFields}{getReprP'"'"'}
\haskellinput{Core/FiniteFields}{charOfP}
Bekanntlich ist $(\_)^p$ auf endlichen Körpern der Charakteristik $p$ 
ein Automorphismus, was das Ziehen von $p$-ten Wurzeln rechtfertigt.
Sicherlich gibt es dazu bessere Algorithmen, jedoch haben wir uns aufgrund der
Einfachheit entschieden, dies in $\F_{p^m}$ durch $(\_)^{p^{m-1}}$ zu
implementieren.
\haskellinput{Core/FiniteFields}{charRootP}

\subsubsection{Instanzen}
Die implementierten Instanzen sind selbstredend gleich denen der Primkörper.

\haskellinput{Core/FiniteFields}{instance .*?=> Eq}
\haskellinput{Core/FiniteFields}{instance .*?=> Show}
\haskellinput{Core/FiniteFields}{instance .*?=> ShowTex}
\haskellinput{Core/FiniteFields}{instance .*?=> Num}
\haskellinput{Core/FiniteFields}{instance .*?=> Fractional}
\haskellinput{Core/FiniteFields}{instance .*?=> NFData}
\haskellinput{Core/FiniteFields}{instance .*?=> Binary}

% vim:set ft=tex foldmethod=marker foldmarker={{{,}}}:
