\section{Endliche Körper}
\subsection{Primkörper}
Die Primkörper werden in dem Modul \texttt{Projekt.Core.PrimeFields}
spezifiziert. Diese werden implementiert als ħIntħ Werte, die durch den Wrapper
ħModħ noch Zusätzlich information enthalten, in welchem Primekörper sich das
Element befindet.

Da wir uns die Charakteristik zu einem solchem Körper auf Typenebene speichern
wollen, führen wir dazu zunächst eine neue Klasse von Datentypen mit dem Namen
ħNumeralħ ein, welche als einzige Funktion 
ħnumValue :: a -> Intħ besitzen.
Diese Funktion soll konstant die Charakteristik wiedergeben.

Nun können wir durch
\haskellinput{Core/PrimeFields}{newtype Mod}
Primkörper definieren, wobei für den Parameter \texttt{n} ein
Datentyp von der Klasse \texttt{Numeral} eingesetzt werden soll.

Um zu einem Element im Primkörper einen Repräsentanten in $\Z$ zu bekommen gibt
es die Funktion
\haskellinput{Core/PrimeFields}{unMod}
Einen Repräsentanten, der nichtnegativ aber kleiner als die Charakteristik ist,
bekommt man durch
\haskellinput{Core/PrimeFields}{getRepr}

Die Instanzen von ħShowħ und ħShowTexħ ermöglicht es, Elemente von Primkörpern
als String oder als Rohes Tex darzustellen.
\haskellinput{Core/PrimeFields}{instance .*?=> Show}
\haskellinput{Core/PrimeFields}{instance .*?=> ShowTex}

Da es sich hier um einen Nichtfreien Datentyp handelt, brauchen wir noch ein
Extensionalitätsaxiom, das beschreibt, wann zwei Elemente gleich sein sollen:
\haskellinput{Core/PrimeFields}{instance .*?=> Eq}

Außerdem wollen wir damit rechnen können und das ganze ist dann auch ein
Endlicher Körper:
\haskellinput{Core/PrimeFields}{instance .*?=> Num}
\haskellinput{Core/PrimeFields}{instance .*?=> FiniteField}
\haskellinput{Core/PrimeFields}{modulus}

Zum bequemen invertieren haben wir auch noch eine Instanz ħFractionalħ
\haskellinput{Core/PrimeFields}{instance .*?=> Fractional}

Weiterhin haben wir noch die folgenden Instanzen:
\haskellinput{Core/PrimeFields}{instance .*?=> Binary}
\haskellinput{Core/PrimeFields}{instance .*?=> NFData}

\subsubsection{Erzeugen von Primkörpern}
Möchte man nun einen Primkörper von beliebiger Charakteristik in einem Haskell
Programm, bietet sich die ħTemplateHaskellħ Funktion ħgenPrimeFieldħ an. Diese
übernimmt das Erzeugen von diversen Instanzen, die nötig sind.
\haskellinput{Core/PrimeFields}{genPrimeField}

Da es sich hierbei um eine Funktion handelt, die per ħTemplateHaskellħ zur
Compilezeit ausgeführt wird, sind zum nutzen zunächst zwei Pragmas nötig:
\begin{hcode}
{-# LANGUAGE QuasiQuotes #-}
{-# LANGUAGE TemplateHaskell #-}
\end{hcode}
Dann kann man sich einen Primkörper der Charakteristik $7$ mit Namen
\texttt{PF} durch die Ziele
\begin{hcode}
$(genPrimeField 7 "PF")
\end{hcode}
erzeugt werden.

\subsection{Erweiterungskörper}

% vim:set ft=tex foldmethod=marker foldmarker={{{,}}}:
