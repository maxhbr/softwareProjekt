\section{Endliche Körper}
\subsection{Prim Körper}
Die Prim Körper werden in der Datei \texttt{Projekt/Core/PrimeFields.hs}
spezifiziert. Das Modul heißt also \texttt{Projekt.Core.PrimeFields}.

Da wir uns die Charakteristik zu einem solchem Körper auf Typenebene speichern
wollen, führen wir dazu zunächst eine neue Klasse von Datentypen mit dem Namen
ħNumeralħ ein, welche als einzige Funktion 
ħnumValue :: a -> Intħ besitzt. Diese Funktion soll konstant 
die Charakteristik wiedergeben.

Nun können wir durch
\haskellinput{Core/PrimeFields}{newtype Mod}
ein Element in einem endichem Körper definieren, wobei für das \texttt{n} ein
Datentyp von der Klasse \texttt{Numeral} eingesetzt wird.


\subsection{Erweiterungskörper}

% vim:set ft=tex foldmethod=marker foldmarker={{{,}}}:
