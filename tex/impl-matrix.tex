\section{Lineare Algebra}
\label{sec:linalg}

\texttt{GalFld/Core/Matrix.hs}

Grundlegende Funktionen der linearen Algebra -- wie man sie im weiteren Verlauf
beispielsweise für den Berlekamp-Algorithmus brauchen wird -- haben wir in der
Datei \url{Core/Matrix.hs} hinterlegt.

Eine Matrix ist dabei der folgende Datentyp:

\haskellinput{Core/Matrix}{data Matrix}

Man hätte auch die Möglichkeit gehabt Matrizen als ħ[[a]]ħ (also als doppelte
Liste) zu implementieren, jedoch haben Listen eine Zugriffszeite von $\cO(l)$ auf
das $l$-te Element und die Abfrage der Länge dauert bei einer Liste der Länge
$n$ $\cO(n)$. ħArraysħ schaffen beides in $\O(1)$, jedoch mit einer
weit größeren Konstante (vgl. \url{}). 

\subsection{Erzeugung von Matrizen und Basisoperationen}

\paragraph{Erzeugung} Entweder erzeugt man eine Matrix direkt als
ħArray (Int,Int) aħ oder durch die Verwendung von ħfromListsMħ.

\haskellinput{Core/Matrix}{fromListsM}

Für den Spezialfall des Vielfachen der Einheitsmatrix kann man auch folgende
Funktion verwenden.

\haskellinput{Core/Matrix}{genDiagM}

\begin{beispiel}
  Möchte man die Matrix $\begin{bmatrix} 2 & 0 \\ 0 & 2 \end{bmatrix} \in
  \Z^{2\times 2}$
  erzeugen, so gibt es die folgenden drei verschiedenen Varianten:
  \begin{enumerate}
    \item ħarray ((1,1),(2,2)) [((1,1),2::Int), ((1,2),2::Int), ((2,1),2::Int), ((2,2),2::Int)]ħ
    \item ħgenDiagM (2::Int) 2ħ
    \item ħfromListsM [[2::Int,0],[0,2]]ħ
  \end{enumerate}
\end{beispiel}

\begin{bemerkung}
  Es gilt anzumerken, dass der Konstruktor für ħArrayħ stets eine
  \emph{vollständige} Liste erwartet. (Vergleiche auch die interne Funktion
  ħgetAllIdxsExceptħ in ħgenDiagMħ.)
\end{bemerkung}

\paragraph{Basisoperationen} 
Selbstredend möchte man eine Matrix auch wieder in Listenform zurückwandeln:

\haskellinput{Core/Matrix}{toListsM}

Die Dimension und Anzahl der Spalten bzw. Zeilen  einer Matrix lässt 
sich durch die Arraydarstellung sehr leicht angeben.

\haskellinput{Core/Matrix}{boundsM}
\haskellinput{Core/Matrix}{getNumRowsM}
\haskellinput{Core/Matrix}{getNumColsM}



Ebenfalls sehr leicht ist ein Test, ob eine quadratische
Matrix vorliegt.

\haskellinput{Core/Matrix}{isQuadraticM}

Ein Element einer Matrix an einer bestimmten Stelle findet man wie folgt.

\haskellinput{Core/Matrix}{atM}

Eine ganze Zeile bzw. Spalte bekommt man durch ħgetRowMħ bzw. ħgetColMħ.

\haskellinput{Core/Matrix}{getRowM}
\haskellinput{Core/Matrix}{getColM}

\paragraph{Aneinanderfügen von Matrizen} 
Wenn man zwei Matrizen horizontal bzw. vertikal aneinanderfügt, erhält man eine
neue Matrix. Dies ist gerade beim Anwenden des Gaußschen Eliminationsverfahrens
von großem Nutzen.

%\haskellinput{Core/Matrix}{(<|>)}
%\haskellinput{Core/Matrix}{(<->)}

\paragraph{Untermatrizen} Untermatrizen erhält man wie folgt.

\haskellinput{Core/Matrix}{subM}
\haskellinput{Core/Matrix}{subArr}

\paragraph{Vertauschen von Zeilen bzw. Spalten} Ebenfalls beim Gauß-Verfahren
von Nöten ist das Vertauschen von Zeilen bzw. Spalten.

\haskellinput{Core/Matrix}{swapRowsM}
\haskellinput{Core/Matrix}{swapColsM}

\haskellinput{Core/Matrix}{swapRowsArr}
\haskellinput{Core/Matrix}{swapColsArr}

\subsection{Zweiwertige Operationen auf Matrizen}

\paragraph{Addition} Die Addition zweier Matrizen erklärt sich von selbst.

\haskellinput{Core/Matrix}{addM}

\paragraph{Multiplikation} Die Multiplikation wurde nach dem Standardverfahren
implementiert.

\haskellinput{Core/Matrix}{multM}

\subsection{Lineare Algebra}

\paragraph{Transponieren} 

\haskellinput{Core/Matrix}{transposeM}

\paragraph{Zeilenstufenform} 
Um eine Matrix in Zeilenstufenform zu bringen, verwenden wir den allseits
bekannten Algorithmus.

\haskellinput{Core/Matrix}{echelonM}

Nahezu selbsterklärend beginnt der Algorithmus mit einer Fallunterscheidung.
Ist das aktuell zu bearbeitende Element $0$ und die gesamte darunterliegende
Spalte auch, so geht es mit ħechelonM'_noPivotħ weiter. Ist der aktuelle 
Eintrag $0$, wird aber ein Pivotelement gefunden, so vertauscht man die Zeilen
passend (ħswapRowsArrħ). Ist der aktuelle Eintrag $\neq 0$, so liefert 
ħechelonM'_Pivotħ den passenden Eliminationsschritt, durch die Anwendung von
ħarrElimħ.

\haskellinput{Core/Matrix}{arrElim}


\begin{beispiel}
  Wir versuchen einmal die ,,Telefonmatrix`` 
  \[ M = \begin{bmatrix} 7&8&9\\ 4&5&6\\ 1&2&3\end{bmatrix} = 
      \begin{bmatrix}2&3&4 \\ 4&0&1\\ 1&2&3\end{bmatrix} \ 
      \in \F_5^{3\times 3}\]
  in Zeilenstufenform zu bringen. Lässt man sich die einzelnen 
  Zwischenschritte von ħechelonMħ ausgeben, so erhält man:
  \begin{hcode}
echelonM' (k,l)=(3,3) m=
2₅ 3₅ 4₅ 
4₅ 0₅ 1₅ 
1₅ 2₅ 3₅ 
	->(1,1)/=0
echeonM'_Pivot m'=
1₅ 4₅ 2₅ 
0₅ 4₅ 3₅ 
0₅ 3₅ 1₅
echelonM' (k,l)=(2,2) m=
4₅ 3₅ 
3₅ 1₅ 
	->(1,1)/=0
echeonM'_Pivot m'=
1₅ 2₅ 
0₅ 0₅ 
  \end{hcode}
  Die eigentliche Ausgabe der Funktion ħechelonMħ ist dann selbstverständlich:
\begin{hcode}
1₅ 4₅ 2₅ 
0₅ 1₅ 2₅ 
0₅ 0₅ 0₅ 
\end{hcode}
\end{beispiel}


\paragraph{Kern} Mit Hilfe der Zeilenstufenform kann man wie folgt den Kern 
einer Matrix berechnen: Fügt man die Einheitsmatrix passender Größe unten an
die ursprüngliche Matrix an, berechnet dann die Spaltenstufenform der
zusammengesetzten Gesamtmatrix, so sind die Nichtnullspalten des unteren Teils
des Ergebnisses gerade eine Basis des Kerns der ursprünglichen Matrix (vgl.
\cite[Abschnitt Basis]{wiki:matrix-kernel}). Dies lässt sich durch
Transponieren natürlich leicht auf die Berechnung einer Zeilenstufenform
zurückführen.

\haskellinput{Core/Matrix}{kernelM}

\paragraph{Determinante} Offensichtlich kann man mit der Zeilenstufenform auch
die Determinante einer Matrix berechnen.

\haskellinput{Core/Matrix}{detM}

Hier wurde der Algorithmus zur Zeilenstufenform nahezu erneut implementiert, um
die konkrete Elimination in den einzelnen Spalten auslassen zu können, die bei
der Berechnung der Determinante ja unnötig ist.

\paragraph{Determinante ohne \lstinline{Fractional}} Bekanntlich lässt sich die
Determinante einer Matrix über jedem Ring definieren. Das bedeutet, dass es
auch ohne die zur Berechnung der Zeilenstufenform nötigen ħFractionalħ-Instanz
geht, was ħdetLapMħ liefert.\footnote{Man hätte auch die Berechnung der
Smithschen-Normalform implementieren können, die ebenfalls ohne
\lstinline{Fractional}
ausgekommen wäre. Jedoch haben wir uns entschieden darauf zu verzichten, da im
vorliegenden Anwendungsfall der endlichen Körper stets Inverse zur Verfügung
stehen.}

\haskellinput{Core/Matrix}{detLapM}


\subsection{Weiteres}

\paragraph{Alle Matrizen mit gewissen Einträgen} Möchte man alle Matrizen
erzeugen, die eine vorgegebene Größe und vorgegebene Einträge besitzen, so
liefert ħgetAllMħ die Antwort.

\haskellinput{Core/Matrix}{getAllM}
