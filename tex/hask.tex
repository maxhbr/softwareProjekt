\section{Über die Programmiersprache}

\subsection{Hello World}

\subsection{Extensions}
\subsection{Template Haskell}

\section{Ausführen von Haskell Programmen}

\subsection{Compiler basierte Optimierung}

\subsection{Paralleles ausführen}

\section{Testing: \texttt{hspec}}
\begin{itemize}
  \item \url{https://hackage.haskell.org/package/hspec}
  \item \url{http://hspec.github.io/}
\end{itemize}
Hspec is roughly based on the Ruby library RSpec. However, Hspec is just a
framework for running HUnit and QuickCheck tests. Compared to other options, it
provides a much nicer syntax that makes tests very easy to read.

\section{Benchmarking: \texttt{criterion}}
\begin{itemize}
  \item \url{https://hackage.haskell.org/package/criterion}
\end{itemize}
This library provides a powerful but simple way to measure software
performance. It provides both a framework for executing and analysing
benchmarks and a set of driver functions that makes it easy to build and run
benchmarks, and to analyse their results.

\section{Zusammenfügen: \texttt{cabal}}
\begin{itemize}
  \item \url{https://hackage.haskell.org/package/Cabal}
  \item \url{http://www.haskell.org/haskellwiki/How_to_write_a_Haskell_program}
\end{itemize}
The Haskell Common Architecture for Building Applications and Libraries: a
framework defining a common interface for authors to more easily build their
Haskell applications in a portable way.

The Haskell Cabal is part of a larger infrastructure for distributing,
organizing, and cataloging Haskell libraries and tools.


\lstinputlisting[ firstline=2
                , lastline=18 ]{../galFld.cabal}

\subsection{Die eigentliche Libary}
\lstinputlisting[ firstline=20
                , lastline=60 ]{../galFld.cabal}

\subsection{Tests}
\lstinputlisting[ firstline=62
                , lastline=69 ]{../galFld.cabal}

\subsection{Benchmarks}
\lstinputlisting[ firstline=94
                , lastline=105 ]{../galFld.cabal}
