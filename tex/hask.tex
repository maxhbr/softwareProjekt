Gute Bücher zum Einstieg in Haskell sind beispielsweiße \cite{Hutton} und
\cite{lyahfgg}. Basierend auf \cite{Hutton} gibt es von Erik Meijer auch eine
ausführliche Video Reihe, die leicht im Internet zu finden ist.

Eine ausführliche Übersicht über Tutorials bietet die Seite
\url{http://www.haskell.org/haskellwiki/Tutorials}.

\section{Über die Programmiersprache}
Die Einleitung in \cite{lyahfgg} sagt folgendes über Haskell.

\begin{itemize}
  \item 
Haskell ist eine \emph{rein funktionale} Programmiersprache.
In einer \emph{imperativen} Programmiersprache gibt man dem Computer eine Folge
von Aufgaben, welche dann ausgeführt werden.  Dazu gibt es Strukturen, die den
Ablauf steuern, wie beispielsweise \texttt{for} und \texttt{while}.

Anders hingegen in einer funktionalen Programmiersprache. Man sagt dem Computer
nicht, was er tun soll. Stattdessen sagt man ihm eher, was die Sachen sind.
Zum Beispiel, kann man dem Computer sagen, dass die Fakultät einer Zahl das
Produkt aller Zahlen von $1$ bis zu dieser Zahl ist.
Dies wird als eine, häufig rekursive, Funktion ausgedrückt.

Beim funktionalen Programmieren kann man keine Werte von Variablen verändern.
In einer rein funktionalen Programmiersprache hat eine Funktion keine
Seiteneffekte. Das einzige was eine Funktion tun kann, ist eine Berechnung,
basierend auf ihren Eingaben. Das scheint eine Einschränkung zu sein, aber in
der Tat hat dies einige positive Konsequenzen. Beispielsweise liefert eine
Funktion bei gleichen Eingaben, unabhängig von der Umgebung, immer den gleichen
Rückgabe Wert. Diese Eigenschaft heißt \emph{Referentielle Transparenz}.

  \item 
Haskell ist \emph{lazy}.
Das bedeutet, dass Haskell Funktionen nicht auswertet, solange das Ergebnis
nicht benötigt wird. Dies wird durch Referentielle Transparenz ermöglicht.
Haskell bemüht sich, die Auswertungen von Ausdrücken so lange wie möglich zu
vermeiden.
Es wird damit auch ermöglicht scheinbar unendliche Datenstrukturen zu
verwenden, da nur Teile, also so weit wie nötig, evaluiert werden.

  \item 
Haskell ist \emph{statisch Typisiert}.
Das bedeutet, dass der Computer bereits zur Compilezeit weiß, welcher Teil eine
Zahl ist und was eine Funktion ist, die aus einer Liste von Zahlen einen String
macht, usw. Das bedeutet, dass viele Fehler bereits wärend des Compilierens
erkannt werden können.

Zusätzlich ist Haskell auch noch sehr gut darin, Typen zu inferieren. Das
bedeutet, das man meist nicht extra angeben muss, welchen Typ jeder Teil im
Code hat. Beispielsweise erkennt Haskell aus \texttt{a = 4 + 5} dass
\texttt{a} eine Zahl sein muss.
Damit ist es auch leichter, allgemeineren Code zu schreiben, der an vielen
Stellen anwendbar ist.

  \item 
Haskell ist \emph{elegant und präzise}.
Da Haskell viele Konzepte höherer Programmiersprachen nutzt, sind in Haskell
geschriebene Programme meist kürzer als ein vergleichbares imperatives. Und
kürzere Programme sind einfacher zu Warten und enthalten weniger Fehler.
\end{itemize}

Die Haskell Entwicklung begann 1987, als sich eine Gruppe von Wissenschaftlern
zusammengetan hat, um eine Programmiersprache zu entwickeln, die ihren
Ansprüchen genügt. Der \emph{Haskell Report}, welcher die erste stabile Version
beschreibt, wurde 1999 publiziert (überarbeitet Version: \cite{haskell98}).
Der aktuelle Standard wird beschrieben in \cite{haskell2010}.

\section{Ausführen von Haskell Programmen}

Haskell kann jederzeit interpretiert oder compiliert werden. Mit dem
Interpreter \texttt{ghci} oder \texttt{hugs} kann man einfach Programme oder
Code-Schnipsel testen.
Alternativ erhält man durch compilieren mit \texttt{ghc} ausführbare Dateien,
welche dank recht umfangreicher Optimierung performanter sind. Für eine
ausführlichere Optimierung gibt es den Compiler Parameter \texttt{-O}. Noch
mehr Optimierung verspricht \texttt{-O2}, wobei das compilieren damit
nochmals deutlich länger dauert.

Die Compileroption \texttt{-threaded} bereitet die ausführbare Datei darauf
vor, parallel ausgeführt zu werden. Zusätzlich muss man beim ausführen dann
noch die Parameter \texttt{-RTS -N4} mitgeben, wobei die $4$ die Anzahl der
Prozessorkerne angibt, die genutzt werden sollen.

\section{Testing: \texttt{hspec}}
\begin{quote}
  Hspec is roughly based on the Ruby library RSpec. However, Hspec is just a
  framework for running HUnit and QuickCheck tests. Compared to other options,
  it provides a much nicer syntax that makes tests very easy to
  read.\footnote{\url{https://hackage.haskell.org/package/hspec}}
\end{quote}
Weitere Quellen:
\begin{itemize}
  \item \url{http://hspec.github.io/}
\end{itemize}   

\section{Benchmarking: \texttt{criterion}}
\begin{quote}
  This library provides a powerful but simple way to measure software
  performance. It provides both a framework for executing and analysing
  benchmarks and a set of driver functions that makes it easy to build and run
  benchmarks, and to analyse their
  results.\footnote{\url{https://hackage.haskell.org/package/criterion}}
\end{quote}

\section{Zusammenfügen: \texttt{cabal}}
\begin{quote}
  The Haskell Common Architecture for Building Applications and Libraries: a
  framework defining a common interface for authors to more easily build their
  Haskell applications in a portable way.

  The Haskell Cabal is part of a larger infrastructure for distributing,
  organizing, and cataloging Haskell libraries and
  tools.\footnote{\url{https://hackage.haskell.org/package/Cabal}}
\end{quote}
Weitere Quellen:
\begin{itemize}
  \item \url{http://www.haskell.org/haskellwiki/How_to_write_a_Haskell_program}
\end{itemize}

\section{Dokumentation: \texttt{haddock}}
\begin{quote}
  Haddock is a tool for automatically generating documentation from annotated
  Haskell source code.\footnote{\url{http://www.haskell.org/haddock/}}
\end{quote}
